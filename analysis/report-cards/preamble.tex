\usepackage{fancyhdr}

\geometry{
  left=3cm,
  right=3cm,
  top=2cm,
  bottom=3cm
}

\fancypagestyle{headeronly}{%
  \fancyhf{}
  \renewcommand{\headrulewidth}{0pt}
  \pagestyle{fancy}
  \setlength{\headheight}{40pt}
  \lhead{\includegraphics[width = .20\textwidth]{../../data/report-cards/logos/VibrantOceansInitiative_BPProgram_Logo_RGB-Color.png}}
  \rhead{\includegraphics[width = .07\textwidth]{../../data/report-cards/logos/WCS-ACRONYM-transp.png}\includegraphics[width = .05\textwidth]{../../data/report-cards/logos/wwf.png}}
}

\fancypagestyle{headerandfooter}{%
  \fancyhf{}
  \renewcommand{\headrulewidth}{0pt}
  \pagestyle{fancy}
  \setlength{\headheight}{40pt}
  \lhead{\includegraphics[width = .20\textwidth]{../../data/report-cards/logos/VibrantOceansInitiative_BPProgram_Logo_RGB-Color.png}}
  \rhead{\includegraphics[width = .07\textwidth]{../../data/report-cards/logos/WCS-ACRONYM-transp.png}\includegraphics[width = .05\textwidth]{../../data/report-cards/logos/wwf.png}}
  \setlength\footskip{11pt}
  \lfoot{\fontsize{9}{9} \selectfont 1) This is a mapping exercise that ranks a series of global data layers to understand the different contexts and/or pressures of various reef locations. Local knowledge and validation is crucial. \\ 2) Preliminary results are subject to change following scientific peer review; updated data layers can be found on https://programs.wcs.org/vibrantoceans/Map}
}

\pagestyle{headeronly}
